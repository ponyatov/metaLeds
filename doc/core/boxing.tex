\clearpage
\secrel{Boxing and no-syntax}\label{boxing}

\lst{doc/core/boxing.py}{py}{\fn{Object.box()}}
\clearpage

This tiny section shows, how to automatically wrap some \py\ types to \mel\
objects. It goes after the Primitives \ref{prim}\ because they were used for the
demo, but the full real code you can see in the \file{metaL.py}.

\medskip
\lst{doc/core/boxing1.py}{py}{operators with boxing}
\clearpage

Boxing is important due to the \mel\ main goal: seamlessly integrate the
\lisp-like homoiconic interpreter engine with any existing source code does not
matter what host language was used for it.

\label{nosyntax}

\mel\ is especially called \term{no-syntax language} because \emph{it relies
entirely on parser and runtime provided by a mature compiler or interpreter}. It
also has some disadvantage: you are forced to use the real syntax of your host
language. So, any program in metaL will look more or less different in every
host syntax.

\py\ is very good or maybe the best as a host language due to its ease of use,
lite interpreter, dynamic nature, multiple inheritance for classes, and
especially a huge amount of ready-to-use libraries.

Boxing lets you write code very very similar to \py\ code because \mel s custom
operators can take \py\ data types as operands, and automatically wrap them into
(hyper)objects. It is so good, that IDEs and PEP8 autoformatter can process your
mixed code without any problems. \note{with other less flexible host languages,
you must call all constructors literally. \ref{ply}}

\clearpage
\lst{doc/core/boxing2.py}{py}{operators with boxing}
