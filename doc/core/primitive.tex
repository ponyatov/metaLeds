\clearpage
\secrel{Primitive types}\label{prim}\secdown

Most programming language descriptions traditionally start with literal elements
such as numbers and strings. In \mel\ primitives not so primitive, as they
inherited from the \class{Object} \ref{object} which is composited by design.
The main property of primitive is it \textbf{evaluates into itself} and
represents a single scalar value.

\bigskip
\lst{doc/core/primitive.py}{py}{\class{Primitive}}

\clearpage
\secrel{Nil}\label{nil}

\lst{doc/core/nil.py}{py}{\class{Nil}}

\secrel{Name}\label{name}

\lst{doc/core/name.py}{py}{\class{Name}}

\clearpage
\secrel{String}\label{string}

\lst{doc/core/string.py}{py}{\class{String}}

\clearpage
\secrel{Number}\label{number}

Practical use of \mel\ shows that in most cases floating point numbers used for
measurement results. An important thing about measurements is values have not
only limited precision, but also can have tolerance ranges, and units. Input
must be checked with ranges, and the user should be able to input "nullable"
values\note{or even have a selection from set of NaN options like "no data",
"invalid measurement",..}. \mel 's data model let us add any attributes
individually or while-class.

\lst{doc/core/number.py}{py}{\class{Number}}

% \clearpage
\lst{doc/core/number1.py}{py}{\fn{Number.val()}}

\secrel{Integer}\label{integer}

\lst{doc/core/integer.py}{py}{\class{Integer}}

\secup
