\clearpage
\secrel{Object (hyper)Graph}\label{og}

Groups of linked \class{Object}s \ref{object}\ form a \term{cyclic directed hypergraph}.

\begin{itemize}
    \item \url{https://en.wikipedia.org/wiki/Hypergraph}
\end{itemize}

When we speak about using \mel's \term{object graph} as a base for some
\term{object/graph database}, it is important to point on the difference between
other graph databases and this \term{hypergraph database}: \emph{there is no
ability to add attributes to edges between nodes}.

On the other side, this \term{object hypergraph} has a big advantage: \emph{this
data model is totally native to any modern programming language}:
\begin{itemize}

    \item objects (frames) and their types can be mapped directly to OOP objects
    and classes correspondingly

    \item edges between nodes are just low-level pointers in machine memory

    \item links in this hypergraph have a few notable properties:
    \begin{enumerate}
        \item \term{object-bound locality}\ --- you can not address any graph
        edges until you don't fetch the \term{origin object} which \term{owns}
        them
        \item \term{unidirectional}\ --- no one object knows about other objects
        which references to him
    \end{enumerate}

    \item every object (hypergraph node) can have arbitrary methods, which
    provides an ability to use hypergraph as a \term{universal program
    representation}:
    \begin{enumerate}
        \item \href{https://en.wikipedia.org/wiki/Homoiconicity}{homoiconicity}
        there is no difference between \textbf{program = data}
        \item subgraph executed as a program can manipulate other programs
        (including itself) as a generic data structure, the same way as any
        other data
    \end{enumerate}

    \item specifics of some data type \emph{must be added by inheriting} base
    \class{Object}\ class and adding/overriding methods for custom semantics
    (see next)

\end{itemize}
