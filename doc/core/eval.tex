\clearpage
\secrel{Functional magic of \lisp}\secdown


Consult \cite[chapter 4]{SICP} for this black magic.

\lst{doc/core/eval.py}{py}{\fn{eval()}}
\lst{doc/core/apply.py}{py}{\fn{apply()}}

\clearpage
\lisp\ is the most known magic language, and at the same time, it is also known
as the scariest syntax language. Unsurprisingly, the bracing paren soup will
easily scare the PHP guy.

\textit{The problem is you don't see the core idea behind the esoteric syntax}.
\lisp\ is so internally elegant because of the principle: it also uses the
executable data structure\ --- any program is self-reflective in
runtime\note{even if you don't use this magic}.

What is really important, any construction can be evaluated as a language
expression, any command computes some value making its work. \lisp\ interpreter
does its work as a tiny set of special functions. In \mel, we are more close to
the modern habits with OOP everywhere, so its \term{EDS-interpreter} is spread
everywhere in methods that does the same.

\clearpage
\secrel{\fn{eval()}}\label{eval}

The more easy part of the magic crystal is the \fn{eval()} function.

In contrast to \py, \fn{eval} takes not string as a parameter, but some
composite data structure\note{like AST \ref{ast}, nested list, parsed JSON or
XML in memory, or hypergraph in our case}, and evaluates it in a given
\term{environment}\note{aka \term{namespace} or \term{context}}.

\secrel{\fn{apply()}}\label{apply}

\secup
